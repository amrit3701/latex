\documentclass{book}
\usepackage{geometry}
 \geometry{ a4paper, total={170mm,257mm}, left=20mm,top=20mm }
\usepackage{sagetex}
\usepackage[utf8]{inputenc}
\usepackage{fancyhdr}
\usepackage{amsmath}
\title{Individual Spread Footing}
\author{Manpreet Kaur, Amritpal Singh}
%---------------------------main document-------------------------------%
\begin{document}
\maketitle
\tableofcontents
\chapter{Individual spread Footing}
\section{Introduction}
Clause $34$ of the Code gives the provisions governing the design of reinforced concrete footings. These provisions are similar to those given by the ACI code. The design of footings in accordance with IS : 456—2000 differs from that by the old code in the following aspects.
\begin{enumerate}
\item Perimeter shear stress must not exceed the allowable value. This aspect was not
given in the old code. But it is similar to the familiar concept of punching shear
stress.

\item Bond-stress-criterion was given in the old code, but it is omitted in the new code.
Instead, development length of footing bars is required to be checked at the sections
where bending moment is critical.

\item 25\% excess pressure on edge of footing was allowed by the old code when a footing
 is eccentrically loaded. This concession is withdrawn by the new Code, thereby
adding to the cost of eccentrically loaded footings.
\end{enumerate}

The Code requires footings to be designed for the following limit states.

\begin{enumerate}
\item Perimeter shear
\item Bending moment
\item Beam shear
\item  Development length of footing bars.
\item Development length of column bars.
\end{enumerate}

\subsection{Types of individual Footings}
Individual spread footings can be either square or rectangular in plan, the area of a rectangular footing with sides A and B being given by,

\begin{equation}
\label{eq:(1.1)}
A\timesB =\frac{p}{q} (cm^2)
\end{equation}

where $P$ is the column load in $kN$ and p denotes the net allowable soil pressure in $kN/cm2$. In this development, self-weight of footing may not be considered. This involves only a small error in that, the weight of the concrete of footing is assumed here to be approximately equal to the weight of the earth displaced by it. Further, it is assumed here that the soil pressure under the footing is uniform. This is a reasonable assumption as discussed elsewhere. The individual spread footings can be of the following types (Fig. 12.1).

With the area of footing known from equation \ref{eq:(1.1)}, dimensions A and B are easily finalise. The only dimension of footing which remains to be known in the depth (D) of the footing. A common way of design of footing is to assume D, rather generously, with a view to reduce steel area as well as to help provide fixity to the column base, in the order to be close to the assumptions made in the frame analysis of superstructure.

\section{Design for Perimeter Shear}
Depth of footing is fixed from the consideration of perimeter shear stress which depends on concrete quality, being independent of types of reinforce steel. For a square footing of uniform depth with a square column of side a (Fig 12.1 a), perimeter shear $tv$ is given by

\begin{equation}
\label{eq:(1.2)}
T_v = \frac{V_u} {{b_0}.d} 
=\frac{1.5 S_p} {4(a+d)d} 
\end{equation}

Where 

\begin{equation}
\label{eq:(1.3)}
\quad S_p = P-p . (a+d)^2
\end{equation}

and $b_0$ = Perimeter of criticle closed section
\par The allowable perimeter shear stress
$\quad{T_a}$  
(clause 31.6.3 of the code) is given by,

Where, $fck$ is to be put in $N/m^2$ .\\
 $ks$ = 1.0 for square columns and also for rectangular with aspect ratio  $\left( \frac{b}{a} \right)$ $\leq {2.0}$. For the condition $T_v$ = $Ta$, equation \ref{eq:(12.2)} and \ref{eq:(12.4)} give,

 \begin{sagesilent}
         var('a,A,d,f,c,k,p')
         z1=(1-((a/A)**2)-(2*(a/A)*(d/A))-(d/A)**2)/((a/A)*(d/A)+(d/A)**2)
         z2=(.067*(f*c*k)**(1/2))/p
 \end{sagesilent}

 \begin{equation}
         \sage{z1}=\sage{z2}=\sage{k}
 \end{equation}

 For a square sloped footing with a square column of side a (Fig. 12.2 b),
 
\begin{equation}
\label{eq:(1.5)}
T_v=\frac{V_u}{b_0.d "}
=\frac{15.0\quad{S_p}}{4(a+d)d"}.\alpha-k(say)
\end{equation}

Assuming $d$ = $\alpha.d$ , the condition $T_v=T_a$  gives,

                                                                       
 \begin{equation}                                                       
         \sage{z1}=\sage{z2}=\sage{k}                                   
 \end{equation}  



$\alpha = 1.0$ for footings of Types ($a$), ($b$) and ($d$) (Fig. 12.1), while $\alpha$ < 1.0 for sloped footings of
Type ($c$). The overall depth of footing is given by,

\begin{sagesilent}
        var('D,d,c,phi')
        z=d+c+phi
\end{sagesilent}

\begin{equation}
        \sage{D}=\sage{z}
\end{equation}
Here, $d$ is regarded as an average value for either steel layer. For sloped footings (Fig. 12.2 $b$), simple geometry gives,

\begin{equation}
\label{eq:(1.8)}
\alpha=\frac{d''}{d}=\frac{\quad{D_m}}{d}+\frac{D-\quad{D_m}}{d}.\frac{\left( 1-\frac{a}{A}-\frac{d}{A}\right)}{\left( 1-\frac{a}{A}\right)}
\end{equation}

Chart 12.1 is developed on the basis of equations (12.5) and (12.7) and it applies to both
uniformly deep and sloped square footings. It can also be used for rectangular footings with rectangular columns by using average values of a and A, provided an equal overhang is left on all sides of column, which requires,

\begin{sagesilent}                                                      
        var('b,a,B,A')                                                
        z=b-a
        z2=B-A                                                    
\end{sagesilent}  

\begin{equation}
        \sage{z}=\sage{z2}
\end{equation}

Solution of numerical examples gives an idea that it is possible to develop thumb rules for fixing depth of footings. It may be noted that there is no dire need of exactness in fixing the value of depth of footing, only it should be more than adequate for the actions imposed on a footing. Table 12.1, based on equation (12.5), is developed for footings of Types ($a$) and ($b$) (Fig.12.1).\\\\
 It gives values of $\frac{D}{A}$ for various practicable values of p. It is seen that, for safety in beam\\ \\shear, these values are  to be increased by 10\% in case of steel types $Fe$ $415$ and $Fe$ $ 500$. For sloped and stepped footings (Types c and d), the depth of footing given by Table 12.1 may be increased by 20\%. The depth at the free end of a footing may be restricted to 15 cm, which is the minimum prescribed by the Code for spread footings.
 
 \section{Design for Moment and Beam Shear} 
 Section 1-1 in Fig. 12.3 is the critical  Section for bending moment. The bending moment
for full width $B$ is given by,

\begin{sagesilent}                                                      
        var('p,B,A,a,k,N,cm')                                                
        z=(p*B*(A-a))/8                                           
\end{sagesilent}  

\begin{equation}
        \label{eq:(1.10)}
        M_{1-1}=\sage{z}(kNcm)
\end{equation}

For footings of uniform depth and also stepped footings, the concrete compression zone is
rectangular and charts of Chapter 2 are used to calculate the required area of steel. But for slopped footings, the concrete compression zone is of a trapezoidal shape and Chart 4.1 of Chapter 4 is to be used for finding steel area. Chart 4.1 can be used for both uniformly deep $(\gamma = 0)$ and sloped footings. The calculated steel area should not be less than the spec1fied minimum steel area (Table 11.4 of Chapter 11) for spread footings which may be regarded as slabs for this purpose. 
Section 2-2 in Fig. 12.3 is the critical section for beam shear. The shear force and moment at section 2-2 for the full width of footing is given by,

\begin{sagesilent}                                                      
        var('p,B,b,A,a,d')                                                
        z=(p*B)*((A-a)/2-d)                                                      
        z2=(p*(B/2))*(((A-a)/2)-d)**2
\end{sagesilent}  

\begin{equation}
        '
        b S_{2-2}=\sage{z}(kN)
\end{equation}

\begin{equation}
        M_{2-2}=\sage{z2}(kNcm)
\end{equation}



\begin{sagesilent}
        var('tau_v,V_u')
        z=V_u/(b*d)
\end{sagesilent}

Beam shear,stress $\sage{tau_v}$ for footings of uniform depth and stepped footings is given by,

\begin{equation}
        \sage{tau_v}=\sage{z}= \frac{{1.5} \times S_{2-2}}{bd}
\end{equation}

Where, $$b = B \text{for uniformally deep footings}$$ 
$$=\alpha \text{for stepped footing (Fig 12.1 d)}$$
For sloped footings, clause 40.1.1 of the Code gives, (-ve sign applies here),

\begin{equation}
        \sage{tau_v}=\frac{\quad{V_u}}{b'd'}-\frac{M_u}{b'd'^2}.tan B
\end{equation}

where b', d' are shown in Fig. 12.3 (a) and  12.3 (b) and $\quad{M_u}$ is the ultimate moment at section 2-2.

\begin{sagesilent}
        var('D_m,D,a,A,d')
        z=D_m+((D-D_m)*(1-(a/A)-2*(d/A))/(1-(a/A)))
        var('phi,c')
        z2=c+phi
        var('b,B')
        z3=b+2*((B-b)/(A-a))*d
        be=var('beta')
        z4=tan(be)
        z5=2*((D-D_m)/(A-a))
        z6=2*(D-D_m)/(A-a-D)
\end{sagesilent}
\begin{equation}
        D'=\sage{z}
\end{equation}

\begin{equation}
        d'=D'-(\sage{z2})
\end{equation}

\begin{equation}
        b'=\sage{z3} \sage{be}
\end{equation}

For Type (c), Fig. 12.1 (c) gives,
\begin{equation}
        \sage{z4}=\sage{z5}
\end{equation}


For Type (b), Fig. 12.1 (b) gives,
\begin{equation}
        \sage{z4}=\sage{z6}
\end{equation}

\begin{sagesilent}
        var('k,tau_a,tau_v,tau_c')
        z=k*tau_c
        var('V_v')
        z2=V_v/(b*d)
\end{sagesilent}

The calculated value of $\sage{tau_v}$ must not exceed the allowable stress $\sage{tau_a}$, as shear reinforcement is just not provided in individual footings for reasons of economy, the same as in solid slabs.The allowable stress in concrete solid slabs $\sage{tau_a}$ is given by,

\begin{equation}
        \sage{tau_a}=\sage{z}
\end{equation}

$\sage{tau_c}$ is given by Table 19 of the Code depending on the steel area provided for moment at the critical section 2-2 (the minimum value of $\sage{tau_c}$ is assumed for $p_t<=0.15$ in Table 19 and

$$k=1.0 \text{ for } D \geq 30 \text{cm}$$
$$k=1.1 \text{ for } D = 25 \text{cm}$$   
$$k=1.2 \text{ for } D = 20 \text{cm}$$
$$k=1.25 \text{ for } D = 17.5 \text{cm}$$   
$$k=1.30 \text{ for } D \leq 15 \text{cm}$$   

Normally, depth of footing given for perimeter shear (Chart 12.1) is more than adequate
to satisfy the requirements of beam shear. But when steel type $Fe$ 415 and $Fe$ 500 are used as reinforcement beam shear may govern the depth of footing. For stepped footings, additional checks for moment and beam shear are required to be made for the portion of the footing of depth $D_m$ (Fig. 12.1 d). When section 1’ - 1’ is the critical section for moment and section 2’ — 2’ is that for beam Shear (Fig. 12.1 d), expressions for moment and shear are given as,

\begin{equation}
        M_{1'-1'}=p.B.\frac{(A-a')^2}{B}
\end{equation}

\begin{equation}                                                        
        S_{2'-2'}=p.B\bigg[\frac{(A-a')^2}{B}-d_m\bigg]                                 
\end{equation}

For finding steel area, Chart 4.1 $(with \gamma = 0)$, may be used, as the concrete compression zone is of a rectangular shape of width equal to B. Shear stress $T_v$ is given by,

\begin{equation}
        \sage{tau_v}=\sage{z2}=\frac{1.5S_{2-2}}{B.d_m}
\end{equation}
and it must not exceed Ta given by equation (12. 16), failing which depth $D_m$ should be suitably increased. Normally $D_m$ = 0.30 $D$ to 0.50 $D$ is kept in stepped footings and perimeter shear stress can be checked to be safe by using first principles.

\section{Development Length of Bars}
Column dowel bars should extend into footings for a distance equal to the development
length (Table 11.3 of Chapter 11) of column bars in compression (or in tension when moment in column is large). With the clause 26.2.2.2 of the Code, column bars can always be adequately anchored in the footing, whatever be the depth of footing. 

\begin{sagesilent}
        z=((A-a)/2)-4                    
\end{sagesilent}

For development length of footing bars, there should be adequate bar length available $(y)$, either straight or bent-up or both measured from the face of column. Referring to Fig. 12.4 (a),for sloped or uniformly deep footings,

\begin{equation}
        y= \sage{z} >L_d (tension)
\end{equation}
where 4.0 $cm$ is taken as clear cover over ends of bars in footings. If equation (12.20) is not  satisfied, there are two ways to tackle this problem :

\begin{enumerate}
\item bend bars up, as shown dotted in Fig. 12.4 (a).
\item choose smaller diameter for bars.
\end{enumerate}

For stepped footings Fig. 12.4 (b), the available straight length of bars beyond the critical
section 1’ - 1’ is,
                                                          
\begin{equation}                                                        
        y'= \frac{(A-a')}{2}-4 >L_d (tension)                                      
\end{equation}

Normally, full steel area required at section 1 — 1 is provided throughout and the steel strength $\sigma_s$ at section 1’ - 1’ may be less than its maximum value of 0.87 $f_y$. The value of $L_d$ (tension) should be calculated for the appropriate value of $\sigma_s$.

\section{Selection of Type of Footings}
Footings of uniform depth (Type $\alpha$), though commonly used in practice for reasons of ease in design and construction, are the costliest. These consume more concrete quantity (about 25\% to 45\%) than that by sloped footings. This type is suitable only for small footings with overall depth being restricted to, say, 30 cm.

For footings of intermediate size, sloped footings with slope starting from $\frac{D}{2}$ away from the edge of column

(Type $b$), are quite suitable. This type is quite economical giving concrete and steel quantities quite reasonable in comparison with other types. This type is easy to design as well as to execute. This type is recommended for most individual footings encountered in buildings with overall depth greater than 30 cm. The depth at the free end of footing may be kept at 15 cm, the specified minimum given by the Code. The depth ($D$) of this type of footing is kept the same as that for footings of uniform depth.

For large-sized footings, sloped footings with the slope starting from the edge of column
(Type $c$) or stepped footings (Type $d$) are preferred to other types, as these give the least quantities for concrete and steel consumption. The stepped footings give the least steel quantity, while the sloped footings (Type $c$), give the least concrete quantity. The depth for these types of footings works out to be about 20\% more than that for footings of uniform depth. Stepped footings are a little cumbersome in construction, while the sloped footings are easier in execution, albeit a little more labour-intensive than the footings of uniform depth.

%---------------------------first example-------------------------------%

\section{Examples}
\textbf{ Example 12.1. Square footing of Uniform Depth (Type a).}\\
\textbf{ Given.}
$$P = 1000kN$$
$$p = .02 kN/cm^2$$
$$a = 40 cm$$
$$f_{ck} = 15 N/mm^2$$
$$f_y = 415 N/mm^2$$
\begin{enumerate}
\item \textbf{Dimensions of footing}\\
        Equation (12.1) gives,
        $$A^2=\frac{1000}{0.02}=50,000 cm^2$$
        $$A=223.6 cm$$
Provided $225 \times 225$ base and
        $$p = \frac{100}{225 \times 225} = 0.01975 kN/cm^2$$
        Table 12.1 gives for p = 0.02, and steel $Fe$ 415,
        $$\frac{D}{A} = \frac{1}{4.5} \text{ or } D = \frac{225}{4.5}=50$$
\item   \textbf{Check for perimeter shear}
        Chart 12.1 gives for,
        $$\frac{0.67\sqrt{15}\times 1.0}{0.01975}=13.14$$
        $$\frac{a}{A}=\frac{40}{225}=.018$$
        $$\frac{d}{A} = 0.18$$
        $$d=0.18 \times 225 = 40.5$$
Equation (12.8) gives with $c = 4.0 cm,\phi=1.2cm$,   
        $$D=40.5+4.0+1.2=45.7 cm$$
        $$D = 50cm \text{ is safe.}$$
\item  \textbf{Design for moment}
Equation (12.10) gives,
$$M_{1-1}=0.01975 \times225(225-40)^1/8=19011kN/cm$$
With rectangular compression zone, Chart (2.2) gives for,
        $$d=50-4.0-1.2=44.8cm$$
        $$k=\frac{1.5\times19011}{1.5\times225\times(44.8)^2}=0.042$$
        $$\mu = 0.0475$$
        $$A_{st}=\frac{0.0475\times225\times44.8\times}{0.87\times415}=199.89cm^2$$
        $$\frac{A_{st}}{A}=\frac{19.89}{2.25}=8.84cm^2/m $$
$\phi 12/12 c/c $ both ways provided giving an area$ = 9.42cm^2/m$

Table 11.4 gives the minimum tension steel area in footings taken as slab,
$$A_{st}(min)=0.12\times50=6.0cm^2/m$$
which is exceeded by that provided.
\item  \textbf{Design for beam shear}
        Equation (12.11 a) gives,
        $$S_{2-2}=0.01975\times225\bigg[\frac{225-40}{2}-44.8\bigg]=212kN$$
        Equation(12.12) gives
\begin{sagesilent}
       var('tau_v,tau_c,tau_a')
\end{sagesilent}
$$\sage{tau_v}=\frac{1.5\times212}{225\times44.8}=0.032kn/cm^2$$
For $$\frac{100A_{s}}{bd}=\frac{100\times9.42}{100\times44.8}=0.21,$$

Table 19 of the Code gives,
$$\sage{tau_c}=0.32N/mm^2$$
$$K = 1.0$$
        $$\sage{tau_a}=\sage{tau_c}=0.032kN/cm^2$$
$$\sage{tau_v}=\sage{tau_a}, D = 50cm \text{ is safe}$$

\item  \textbf{Check on development length of footing bars}

Table (11.13) gives, for footings bars of $\phi$ 12 ($F_e$ 415),
$L_d$ (tension) = 55$\phi$ = 55 x 1.2 = 66 cm.
Equation (12.20) gives, 

$$y=\bigg[\frac{(225-40)}{2}-4.0\bigg]=88.5cm$$

With $y > L_d$ (tension), footing bars will develop full strength at the critical section. It may be noted that with $D$ = 50 cm, this type of footing of uniform depth is not economical. Footing of Types ($b$) and ($c$) with sloping depth would be more economical than the present design.\\
\end{enumerate} 

%---------------------------second example-------------------------------%

\textbf{ Example 12.2. Rectangular sloped footing of Type ($c$).}\\
\textbf{Given:}
Same as in Ex. 12.1 with\\
$$a = 40 cm$$
$$b = 60 cm$$ 
and the footing is to have equal overhangs on all sides of column.\\
\textbf{Required:} Design the footing\\
\textbf{Solution.}
\begin{enumerate}
\item   \textbf{Dimensions of footing}\\
  Equation (12.1) gives,
  $$ A\times B= \frac{1000}{0.02}=50000 cm^2$$
  The equal overhang condition, equation (12.9 $a$) gives,
  $$ b-a=B-A=20 cm$$
  The solution of these two equations gives,
  $$A=214 cm B=234 cm$$
  Practical designer may choose A = 215 cm and B: 235 cm
  $$p=\frac{1000}{215\times235}=0.0198  kN/cm^2$$
  Table 12.1 gives for $p$ = 0.02 and steel $Fe$ 415,
 $$\frac{D}{A}=\frac{1}{4.5}\times1.20$$
 $$D=\frac{(215+23)}{2 \times 4.5} \times 1.20=60cm$$
 $$D=60cm and D_m=15cm$$
 
\item  \textbf{Check for perimeter shear}\\
 Equation (12.8) gives, with c = 4.0 cm, $\alpha$ = 1.2 cm,
 $$d=60-(4.0+1.2)=54.8cm$$
 Equation (12.9) gives,
 $$ \alpha=\frac{15}{54.8}+\frac{(60-15)}{54.8}.\frac{\Bigg(1-\frac{40}{215}-\frac{54.8}{215}\Bigg)}{\Bigg(1-\frac{40}{215}\Bigg)}-\frac{(4.0+1.2)}{54.8}$$
 $$=\frac{1}{54.8}(15+30.91-5.2)=0.74$$
 Chart 12.1 gives for,
 $$ k=\frac{0.067\sqrt{15}\times0.74}{0.0198}=9.70$$
 $$\frac{a}{A}(average value)=\frac{(43+60)/2}{(215+285)/2}=\frac{50}{225}=0.22$$
 $$\frac{d}{A}=0.195 or d=0.195\times225=43.9 cm$$
 $$D=43.9+4.0+1.2=49.1 cm$$
 $$D=60cm is safe in perimeter shear.$$
 
\item  \textbf{Design for moment}\\
   Equation (12.10) gives,
   $$M_1-1=0.0198\times235(215-40)^2/8=17812 kN cm$$
   With trapezoidal compression zone, Chart 4.1 gives for,
   $$b=60cm, d=5\fr4.8 cm$$
   $$tan\beta =\frac{(D-D_m)}{(B-b)/2}=\frac{45}{87.5}=0.514$$
   $$\gamma=\frac{54.8}{60\times0.514}=1.8$$
   $$k=\frac{1.5\times17812}{1.5\times60\time(54.8)^2}=0.099$$
   $$\mu=0.11$$
   $$A_st=\frac{0.11\times60\times54.8\times15}{0.87\times415}=15.03cm^2$$
   $$\frac{A_st}{B}=\frac{15.03}{2.85}=640 cm^2/m$$
  Minimum steel for average value of $D =\frac{(60+15)}{2}=37.5$ regarding footings as slabs,
  $$A_st(min)=0.12\times37.5=4.20 cm^2/m$$
  Provide $\phi$ $12/17$ $c/c$ bothways, area=6.65 cm^2/m

\item  \textbf{Design for beam shear}
  Equation (12.11 $a$)gives
  $$S_2-2=0.0198\times235\Bigg[\frac{(215-40)}{2}-54.8\Bigg]=152 kN$$
  Equation (12.14 $a$)gives
  $$D'=15+(60-15)\frac{\Bigg(1-\frac{40}{215}-2\times \frac{54.8}{215}\Bigg)}{\Bigg(1-\frac{40}{215}\Bigg)}$$
  $$=15+16.8=31.8 cm$$
  Equation (12.14 $b$) gives,
  $$ d'=31.8-5.2=26.6 cm$$
  Equation (12.14 $c$) gives,
  $$b'=b+2\frac{(B-b)}{(A-a)}\times d=65+2\times \frac{175}{175}\times 54.8=169.6 cm$$
  Equation(12.11 $b$) gives,
  $$M_2-2=0.0198\times 235(87.6-54.8)^2/2=2488 kN cm$$ 
  Equation(12.13) gives,
  $$T_v=\frac{1.5\times 152}{169.6\times26.6}-\frac{1.5\times 2488}{169.6\times (26.6)^2}\times 0.514$$
  $$=0.050-0.016=0.034 kN cm^2$$
  For
  $$\frac{100A_s}{b'd'}\frac{100\times6.65}{100\times26.6}=0.25,$$
  Table 19 gives to = $0.035 kN/cm2$
  $$k=1.0, T_a=T_c=0.035 kN/cm^2>T_u=0.034 kN/cm^2,$$
$$D=60 cm is safe in beam shear$$  
  
\item  \textbf{Check on development length of footing bars}\\
Table(11.3)gives for $\phi$ 12 bars
$$L_d(tension)=55\times1.2=66cm$$
Chart 12.1. Effective Depth (($d$) of Square Individual Footings for Safety in perimeter shear.
$$k=\frac{\Bigg[1-\left(\frac{a}{A}\right)^2
-2\left(\frac{a}{A}\right)^2\left( \frac{d}{A}\right)-\left(\frac{d}{A}\right)^2\Bigg]}{\Bigg[\left(\frac{a}{A}\right)\left(\frac{d}{A}\right)+\left(\frac{d}{A}\right)^2\Bigg]}$$

\textbf{Notes}
\begin{enumerate}
\item  $fck$ in $N/mm^2$
\item $p$ in $kN/cm^2$b
\item For rectangular column $a\timesb$
Assume $a=\frac{(a+b)}{2}$ as in approximation provided $a/b$ <= $0.50.$
\item Chart can be used for squarish rectangular footings provided an equal overhang is left beyond faces of column with $A=\frac{A+B}{2}$\\
For equal overhang $(b-a)=(B-A)$
\item $\alpha=1.0$ for types (a),(b) and (d)(Fig 12.1)
\item $\alpha<1.0$ for type (c). Use equation(12.9)
\item $\alpha=\frac{D_m}{d}+\frac{D-D_m}{d}.\frac{\left(1-\frac{a}{A}-\frac{d}{A}\right)}{\left(1-\frac{a}{A}\right)}-\frac{(c+\phi)}{d}$

\end{enumerate}
%---------------------------sage graph-------------------------------%

 \begin{sagesilent}                                                      
        var('a,A,d,k')                                                  
        f(k)=(1-(a**2)-(2*(a**2)*(d))-(d**2))/((a*d)+d**2)              
        solve(f(k) == k,d)                                              
        f(d) = -1/2*(2*a^2 + a*k - sqrt(4*a^4 + a^2*k^2 - 4*a^2 + 4*(a^3 - a^2 + 1)*k + 4))/(k + 1)
        z1=plot(f(a=.05),k,0,100,ymin=0,ymax=.30,figsize=(6.5,3),color=("green"),   legend_label= "a/A = 0.05")
        z2=plot(f(a=.1),k,0,100,ymin=0,ymax=.30,figsize=(6.5,3),color=("red"),      legend_label= "a/A = 0.10")
        z3=plot(f(a=.15),k,0,100,ymin=0,ymax=.30,figsize=(6.5,3),color=("yellow"),  legend_label= "a/A = 0.15")
        z4=plot(f(a=.20),k,0,100,ymin=0,ymax=.30,figsize=(6.5,3),color=("blue"),    legend_label= "a/A = 0.20")
        z5=plot(f(a=.25),k,0,100,ymin=0,ymax=.30,figsize=(6.5,3),color=("brown"),   legend_label= "a/A = 0.25")
        z6=plot(f(a=.30),k,0,100,ymin=0,ymax=.30,figsize=(6.5,3),color=("orange"),  legend_label= "a/A = 0.30")
        zz=z1+z2+z3+z4+z5+z6                                            
                                                                        
\end{sagesilent}                                                        
                                                                        
\begin{figure}                                                          
\begin{center}                                                          
        \sageplot{zz,axes_labels=['$k$','$d/A$']}                       
\end{center}                                                            
\end{figure}    

\end{enumerate}
\section{Conclusion}
Provisions of the Code on footings have been applied to individual spread footings, square, or rectangular in plan with depth uniform, varying or stepped. Design aids are given for fixing depth of footings and checking it in respect of requirements of safety in perimeter shear.Procedure for design of footings for bending moment, shear and development length of tension steel bars is given in detail and examples are given to illustrate it. For a large-sized rectangular footing, a footing beam in the long direction will be more economical than the traditional isolated footing. Also, for large square footings, two cross footing beams with a uniform base slab will make for economy.

%---------------------------table -------------------------------%

\begin{table}
\centering
\label{Table 12.1}
\caption{Depth of Footing for type (a) and (b) Fig: 12.1 values for $\frac{D}{A}$}
\begin{tabular}{ |p{3cm}||p{3cm}|p{3cm}|}
 \hline
 $p(kN cm^2)$ & $Fe 250$ & $Fe 415, Fe 500$\\[0.8 ex]
 \hline
 0.005 &$\frac{1}{8.0}$ &$\frac{1}{7.0}$\\[0.5 ex]
 0.010 &$\frac{1}{6.0}$ &$\frac{1}{5.5}$\\[0.5 ex]
 0.015&$\frac{1}{5.5}$ &$\frac{1}{5.0}$\\[0.5 ex]
 0.020 &$\frac{1}{5.0}$ &$\frac{1}{4.5}$\\[0.5 ex]
 0.025 &$\frac{1}{4.5}$ &$\frac{1}{4.0}$\\[0.5 ex]
 0.030 &$\frac{1}{4.0}$ &$\frac{1}{3.5}$\\[2 ex]
 \hline
  \end{tabular}
\end{table}
\textbf{Note:}
\begin{enumerate}
\item \textbf{'A' is the average of sides of rectangular footing.
\item For sloped (type c) and stepped (type d) footings, increased depth given by
the above table by 20\%}.
\end{enumerate}
\end{document}   


